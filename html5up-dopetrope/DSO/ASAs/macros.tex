% !TEX root = main.tex
\usepackage{graphicx,amsmath,amssymb,xspace,bm,url,verbatim}  %natbib


\newif\ifauthor % give author names
\newif\ifsubmission   %for submission version
\newif\iffinal        %for springer final version
\newif\iffull         %for full version
\newif\iftwocol       %for acm ccs two-column format
\newif\ifspringer     %for springer version
\newif\ifnofootnotes  %for springer version
\newif\ifshownotes      %have notes to one another show up

\twocoltrue
\springerfalse

\ifspringer\else

\newtheorem{thm}{Theorem}
\newtheorem{lem}[thm]{Lemma}
\newtheorem{cor}[thm]{Corollary}
\newtheorem{propo}[thm]{Proposition}
\newtheorem{clm}[thm]{Claim}
\newtheorem{defn}[thm]{Definition}

\newtheorem{dup}{Theorem}
\newenvironment{duplicate}{\begin{dup}}{\end{dup}}

\newenvironment{theorem}{\begin{thm}}{\end{thm}}
\newenvironment{lemma}{\begin{lem}}{\end{lem}}
\newenvironment{corollary}{\begin{cor}}{\end{cor}}
\newenvironment{proposition}{\begin{propo}}{\end{propo}}
\newenvironment{definition}{\begin{defn}}{\end{defn}}
\newenvironment{claim}{\begin{clm}\begin{rm}}{\end{rm}\end{clm}}

\iftwocol\else
% deluxe proof environment
\def\qsym{\vrule width0.6ex height1em depth0ex}
\newcount\proofqeded
\newcount\proofended
\def\qed{{\hspace{1pt}\rule[-1pt]{3pt}{9pt}}
\end{rm}\addtolength{\parskip}{-0pt}
\setlength{\parindent}{\saveparindent}
\global\advance\proofqeded by 1 }
\def\qedenv{
\end{rm}\addtolength{\parskip}{-0pt}
\setlength{\parindent}{\saveparindent}
\global\advance\proofqeded by 1 }
\newenvironment{proof}%
 {\proofstart}%
 {\ifnum\proofqeded=\proofended~\qed\fi \global\advance\proofended by 1
  \medskip}
\newenvironment{proofenv}%
 {\proofenvstart}%
 {\ifnum\proofqeded=\proofended\qedenv\fi \global\advance\proofended by 1
  \medskip}
\makeatletter
\def\proofstart{\@ifnextchar[{\@oprf}{\@nprf}}
\def\proofenvstart{\@ifnextchar[{\@osprf}{\@nsprf}}
\def\@oprf[#1]{\begin{rm}\protect\vspace{6pt}\noindent{\bf Proof of #1:\ }%
\addtolength{\parskip}{5pt}\setlength{\parindent}{0pt}}
\def\@osprf[#1]{\begin{rm}\protect\vspace{6pt}\noindent
\addtolength{\parskip}{5pt}\setlength{\parindent}{0pt}}
\def\@nprf{\begin{rm}\protect\vspace{6pt}\noindent{\bf Proof:\ }%
\addtolength{\parskip}{5pt}\setlength{\parindent}{0pt}}
\def\@nsprf{\begin{rm}\protect\vspace{6pt}\noindent%
\addtolength{\parskip}{5pt}\setlength{\parindent}{0pt}}
\fi


\fi

\usepackage{latexsym}   
\usepackage{mathrsfs}   
%\usepackage{eufrak}
\usepackage{mathtools}  %for \xmapsto
\usepackage{color} 
\usepackage{setspace}
\usepackage{xspace}
\usepackage{multirow}
\usepackage{epsfig}

%\newenvironment{frcseries}{\fontfamily{frc}\selectfont}{}
%\newcommand{\textfrc}[1]{{\frcseries#1}}
%\newcommand{\mathfrc}[1]{{\text{\textfrc{#1}}}}

\DeclareMathAlphabet{\mathsl}{OT1}{cmr}{m}{sl}
\DeclareMathAlphabet{\mathsc}{OT1}{cmr}{m}{sc}
\DeclareMathAlphabet{\mathslbf}{OT1}{cmr}{bx}{sl}
% math script font; extra commands to make slightly larger
\DeclareFontFamily{OT1}{pzc}{}
\DeclareFontShape{OT1}{pzc}{m}{it}%
             {<-> s * [1.150] pzcmi7t}{}
\DeclareMathAlphabet{\mathscript}{OT1}{pzc}{m}{it}

%
% hack to reduce spacing in the bibliography (found on-line)
% 
 \let\oldthebibliography=\thebibliography
  \let\endoldthebibliography=\endthebibliography
  \renewenvironment{thebibliography}[1]{%
    \begin{oldthebibliography}{#1}%
      \setlength{\parskip}{0ex}%
      \setlength{\itemsep}{0.5ex}%
  }%
  {%
    \end{oldthebibliography}%
  }


\newcommand{\CondProb}[2]{\Pr[#1\,|\,#2]}

\newenvironment{newmath}{\begin{displaymath}%
\setlength{\abovedisplayskip}{5pt}%
\setlength{\belowdisplayskip}{5pt}%
\setlength{\abovedisplayshortskip}{5pt}%
\setlength{\belowdisplayshortskip}{5pt} }{\end{displaymath}}

\newenvironment{newequation}{\begin{equation}%
\setlength{\abovedisplayskip}{5pt}%
\setlength{\belowdisplayskip}{5pt}%
\setlength{\abovedisplayshortskip}{5pt}%
\setlength{\belowdisplayshortskip}{5pt} }{\end{equation}}

\newcommand{\eqq}{\:=\:}
\newcommand{\leqs}{\:\leq\:}

\setlength{\fboxsep}{3pt}

%\makeatletter
%\renewcommand\bibsection%
%{
%  \section*{\refname
%    \@mkboth{\MakeUppercase{\refname}}{\MakeUppercase{\refname}}}
%}
%\makeatother

\DeclareMathAlphabet{\mathsl}{OT1}{cmr}{m}{sl} 
\DeclareMathAlphabet{\mathsc}{OT1}{cmr}{m}{sc}

\newcounter{ctr}
\newcounter{savectr}
\newcounter{ectr}
\newcounter{eectr}

%Random oracle macros
\newcommand{\roproc}{\mathsc{Hash}}
\newcommand{\roprocsim}{\mathsc{HashSim}}
\newcommand{\rotable}{\mathsf{H}}
\newcommand{\phasethree}{\mathtt{ro}}

\newenvironment{newenum}{%
\begin{list}{{\rm (\arabic{ctr})}\hfill}{\usecounter{ctr} \labelwidth=15pt%
\labelsep=7pt \leftmargin=22pt \topsep=0pt%
\setlength{\listparindent}{\saveparindent}%
\setlength{\parsep}{\saveparskip}%
\setlength{\itemsep}{2pt} }}{\end{list}}

\newenvironment{newenumbf}{%
\begin{list}{{\bf \arabic{ctr}.}\hfill}{\usecounter{ctr} \labelwidth=15pt%
\labelsep=5pt \leftmargin=20pt \topsep=2pt%
\setlength{\listparindent}{\saveparindent}%
\setlength{\parsep}{\saveparskip}%
\setlength{\itemsep}{2pt} }}{\end{list}}

\newenvironment{tiret}{%
\begin{list}{\hspace{1pt}\rule[0.5ex]{5pt}{1pt}\hfill}{\labelwidth=12pt%
\labelsep=2pt \leftmargin=14pt \topsep=1pt%
\setlength{\listparindent}{\saveparindent}%
\setlength{\parsep}{\saveparskip}%
\setlength{\itemsep}{1pt} }}{\end{list}}


\newcommand{\Prob}[1]{{\Pr\left[\,{#1}\,\right]}}
% \newcommand{\CondProb}[2]{{\Pr}\left[\, #1\,\left|\right.\,#2\,\right]}

\newcommand{\secure}{ secure\xspace} %with or without the dash?
\newcommand{\security}{ security\xspace} %with or without the dash?

\newcommand{\bits}{\{0,1\}}
\newcommand{\getsr}{{\:{\leftarrow{\hspace*{-3pt}\raisebox{.75pt}{$\scriptscriptstyle\$$}}}\:}}
\newcommand{\getss}{\gets}

\newcommand{\somemark}{\vartriangleright}
\newtheorem{poi}{$\somemark$}
\newenvironment{point}{\begin{poi}
\begin{rm}}  {\end{rm}
\end{poi}}

\newcommand{\captionfont}{\iftwocol\else \small\fi}
\newcommand{\info}{\phi}         % the string 
\newcommand{\information}{\Phi}  % reveal function (leakage function) -- that which one expect to leak

\newcommand{\heading}[1]{\vspace{4pt}\noindent\underline{\textsc{#1}}}
\newcommand{\uheading}[1]{\vspace{5pt}\noindent\underline{#1}}
\newcommand{\headingg}[1]{\noindent\underline{\textsc{#1}}}
\newcommand{\noskipheading}[1]{\iftwocol\noindent\fi \textsc{#1} }



\newcommand{\smidge}{\hspace{0.1ex}}
\newcommand{\smidgee}{\hspace{0.2ex}}


\ifshownotes
\newcommand{\prnote}[1]{{\footnotesize \bf PR writes: \sl #1}}
\newcommand{\mbnote}[1]{{\footnotesize \bf MB writes: \sl #1}}
\newcommand{\kpnote}[1]{{\footnotesize \bf KP writes: \sl #1}}
\else
\newcommand{\prnote}[1]{}
\newcommand{\mbnote}[1]{}
\newcommand{\kpnote}[1]{}
\fi



\newcommand{\expReal}{\mathrm{Real}}
\newcommand{\expFake}{\mathrm{Fake}}


\newcommand{\vecx}{\mathbf{x}}
\newcommand{\vecs}{\mathbf{s}}
\newcommand{\vect}{\mathbf{t}}
\newcommand{\vecTw}{\mathbf{T}}
\newcommand{\vecy}{\mathbf{y}}
\newcommand{\vecz}{\mathbf{z}}
\newcommand{\vectau}{\bm{\tau}}
\newcommand{\dotdot}{\,..\,}

\newcommand{\Initialize}{\mathsc{Initialize}}
\newcommand{\Ch}{\mathsc{Ch}}
\newcommand{\Finalize}{\mathsc{Finalize}}

\newcommand{\Cipher}{\mathsc{Enc}}
\newcommand{\DCipher}{\mathsc{Dec}}
\newcommand{\PProc}{\mathsc{\Pi}}
\newcommand{\PInvProc}{\mathsc{\Pi^{-1}}}



\newcommand{\siminput}{\mathsl{input}}
\newcommand{\query}{\mathsl{query}}
\newcommand{\xxx}{\ensuremath{\mathrm{xxx}}}
\newcommand{\counter}{\mathsl{cnt}}

\newcommand{\AES}{\mathrm{AES}}
\newcommand{\sha}{\mathrm{sha}}
\newcommand{\xor}{\oplus}
\newcommand{\xxor}{\!\oplus\!}
\newcommand{\eq}{\!=\!}
\newcommand{\gatefield}{\gamma} 

\newcommand{\firstlast}[1]{#1}
\newcommand{\blocklen}{k}
\newcommand{\tlen}{\tau}
\newcommand{\klen}{k}
\newcommand{\nlen}{\eta}

\newcommand{\ivl}{\eta}   %IV length

\newcommand{\AD}{A}
\newcommand{\IV}{N}
\newcommand{\IVr}{\mathit{IV}}
\newcommand{\ADspace}{{\mathcal{A}}}
\newcommand{\IVspace}{{\mathcal{N}}}
\newcommand{\calIV}{{\IVspace}}
\newcommand{\calAD}{{\ADspace}} 

\newcommand{\cross}{\times}
\newcommand{\apref}[1]{Appendix~\ref{#1}}
\newcommand{\corref}[1]{Corollary~\ref{#1}}
\newcommand{\figref}[1]{Fig.~\ref{#1}}
\newcommand{\lemref}[1]{Lemma~\ref{#1}}
\newcommand{\secref}[1]{Section~\ref{#1}}
\newcommand{\thref}[1]{Theorem~\ref{#1}}
\newcommand{\defref}[1]{Definition~\ref{#1}}
\newcommand{\propref}[1]{Proposition~\ref{#1}}

\ifspringer
\newtheorem{newdef}[theorem]{Definition}
\newtheorem{newprop}[theorem]{Proposition}
\fi

\iftwocol
%\newtheorem{theorem}{Theorem}
\newtheorem{newprop}{Proposition}
%\newtheorem{corollary}{Corollary}
%\newtheorem{lemma}{Lemma}
\fi

\newcommand{\hF}{{E}}
\newlength{\saveparindent}
\setlength{\saveparindent}{\parindent}
\newlength{\saveparskip}
\setlength{\saveparskip}{\parskip}

\newcommand{\Colon}{\!:\!}
\renewcommand{\ng}{q}   %number of gates  
\newcommand{\nq}{q}   
\newcommand{\nw}{r}     %number of wires 
\newcommand{\nr}{r}    
\newcommand{\bt}[1]{T^{#1}_{0}}

\newcommand{\out}{0}

\newcommand{\BOT}{\varepsilon}
\newcommand{\emptystring}{\varepsilon}

\newcommand{\Real}{\mathrm{Real}}
\newcommand{\Fake}{\mathrm{Fake}}

\newcommand{\incomingleft}{A}
\newcommand{\incomingleftt}[1]{\incomingleft_{#1}}
\newcommand{\incomingright}{B} 
\newcommand{\incomingrightt}[1]{\incomingright_{#1}}
\newcommand{\outgoing}{C} 
\newcommand{\outgoingg}[1]{\outgoing_{#1}}
\newcommand{\outputtowire}{w}    %was omega D Z  W
\newcommand{\outputtowiree}[1]{\outputtowire_{#1}}
\newcommand{\function}{G}   %f or g or F or G?
\newcommand{\functionn}[1]{\functionn_{#1}}
\newcommand{\outputs}{\Rightarrow}

\newcommand{\concat}{\;\|\;}
\newcommand{\concatt}{\,\|\,}
\newcommand{\concattt}{\:\|\:}
\newcommand{\Concat}{}
\newcommand{\Concatt}[2]{{#1}\concat{#2}}

\newcommand{\gate}{g} % a generic gate
\newcommand{\gatee}{h} % a generic gate

\newcommand{\success}{\textsf{success}}
\newcommand{\outoftime}{\textsf{out-of-time}}

\newcommand{\seed}{S}
\newcommand{\triples}{\mathsf{Tr}}
\newcommand{\set}[2]{\{#1\!:\,#2\}}
\newcommand{\graph}{\mathsf{G}}
\newcommand{\El}{\mathsf{El}}

\newcommand{\BB}{\mathsf{A}}
\newcommand{\BBsq}{\mathsf{A}_{s,q}}
\newcommand{\BBkl}{\BB.\mathsf{kl}}
\newcommand{\BBsqkl}{\BBsq.\mathsf{kl}}
\newcommand{\calA}{{\mathcal{A}}}
\newcommand{\calB}{{\mathcal{B}}}
\newcommand{\calC}{{\mathcal{C}}}
\newcommand{\calD}{\SE.\mathsf{Dec}}
\newcommand{\calE}{\SE.\mathsf{Enc}}
\newcommand{\calF}{{\mathcal{F}}}
\newcommand{\calI}{{\mathcal{I}}}
\newcommand{\calK}{\mathcal{K}}
\newcommand{\calM}{{\mathcal{M}}}
\newcommand{\calS}{{\mathcal{S}}}
\newcommand{\calT}{{\mathcal{T}}}
\newcommand{\calU}{{\mathscr{U}}}
\newcommand{\calX}{{\mathcal{X}}}
\newcommand{\calY}{{\mathcal{Y}}}
\newcommand{\calZ}{{\mathcal{Z}}}


\newcommand{\FF}{\calF}  %{\mathrm{Functs}}}
\newcommand{\sep}{,\,}  
\newcommand{\sepFF}{,\,\FF}  %{\mathrm{Functs}}}
\newcommand{\N}{{\mathbb{N}}}
\newcommand{\R}{{\mathbb{R}}}
\newcommand{\Z}{{\mathbb{Z}}}

%\newcommand{\GCEval}{\mathsf{Eval}}
\newcommand{\GC}{F}  %\textsl{GC}}
\newcommand{\indcomp}{\stackrel{c}{\approx}}

\newcommand{\tobits}[1]{\langle #1\rangle}
% \newcommand{\code}[1]{\langle {#1} \rangle}
\newcommand{\code}[1]{({#1})}

\newcommand{\ttt}{t}
\newcommand{\sss}{s}
\newcommand{\SSS}{\{0,1\}^*}


\newcommand{\Outputs}{\mathrm{Outputs}}
\newcommand{\OutputWires}{\mathrm{OutputWires}}
\newcommand{\InputWires}{\mathrm{InputWires}}
\newcommand{\Inputs}{\mathrm{Inputs}}
\newcommand{\Wires}{\mathrm{Wires}}
\newcommand{\Outgoing}{\mathrm{Outgoing}}
\newcommand{\Gates}{\mathrm{Gates}}
\newcommand{\Table}{\mathrm{Table}}
\newcommand{\Tables}{\mathrm{Tables}}
\newcommand{\Token}{\mathrm{Token}}
\newcommand{\Tokens}{\mathrm{Tokens}}
\newcommand{\Params}{\mathrm{Params}}

\newcommand{\Prop}{\mathsl{Prop}}
\newcommand{\propagate}{P}   %was p

\newcommand{\entry}[1] {{{\propagate}_{#1}}}
\newcommand{\entryy}[2]{{{\propagate}[{#1},{#2}]}} 

\newcommand{\Entry}[1] {{\entry{#1}}}
\newcommand{\Entryy}[2]{{\Entryy{#1}{#2}}}

\newcommand{\row}[1]{\mathsl{row}[{#1}]}

\newcommand{\interpretation}{S} % make it consistent with Garble

\newcommand{\atoken}{V}
\newcommand{\btoken}{I}
\newcommand{\tokenlist}{X}   % list of n tokens, one for each input 
\newcommand{\tokenlistt}{T}  % list of 2n tokens, two for each input
\newcommand{\tokenlisttt}{Z} % list of r tokens, one for each wire


\newcommand{\token}{X}

\newcommand{\GCGoutput}{(\CCC, \E, \D)}   

\newcommand{\Garbleoutput}{(\F,\E, \D)}
\newcommand{\Simoutput}   {(\F, \tokenlist, \D)}

\newcommand{\vecT}{{T}}
\newcommand{\tokenotp}{\vecT}    %{\cal T}}
\newcommand{\tokeninput}{x}
\newcommand{\closure}{\calX}
\newcommand{\closuree}{\closure}
\newcommand{\bitinput}{x}  %X
\newcommand{\bitclosure}{\calX}
\newcommand{\bitclosuree}{\bitclosure}
\newcommand{\CC}{\f}
%\newcommand{\CCC}{\calC}
\newcommand{\CCC}{\F}
\newcommand{\TopoC}{f^{-}}

\newcommand{\ia}{{a}}  %was alpha, beta, gamma
\newcommand{\ib}{{b}}
\newcommand{\ic}{{c}}

\newcommand{\na}{{\mathtt{\alpha}}}
\newcommand{\nb}{{\mathtt{\beta}}} 
\newcommand{\nc}{{\mathtt{\varsigma}}}  %zeta
\newcommand{\np}{{\mathtt{P}}} 

\newcommand{\tta}{{\mathtt{a}}}
\newcommand{\ttb}{{\mathtt{b}}}
\newcommand{\ttc}{{\mathtt{c}}}
\newcommand{\ttx}{{\mathtt{x}}}

\newcommand{\rma}{{\mathrrm{a}}}
\newcommand{\rmb}{{\mathrm{b}}}
\newcommand{\rmc}{{\mathrm{c}}}

\newcommand{\nK}{K}
\newcommand{\nA}{A}
\newcommand{\nL}{{\mathtt{L}}}
\newcommand{\nR}{{\mathtt{R}}}
\newcommand{\nB}{B}
\newcommand{\nC}{{\mathtt{C}}} 
\newcommand{\nT}{T} 
\newcommand{\nV}{{\mathtt{V}}}
\newcommand{\nI}{{\mathtt{I}}}
\newcommand{\ttA}{\nA}
\newcommand{\ttB}{\nB}
\newcommand{\ttK}{K}
\newcommand{\ttP}{{\mathtt{P}}}
\newcommand{\ttC}{{\mathtt{C}}}
\newcommand{\ttS}{{\mathtt{S}}}
\newcommand{\ttT}{\nT}
\newcommand{\ttX}{\mathtt{X}}
\newcommand{\droplsb}[1]{#1_*}

\newcommand{\alphaa}{\gamma''}
\newcommand{\betaa}{\gamma'}

\newcommand{\PPP}{F} 

\newcommand{\then}{;\;}
\newcommand{\andthen}{\!:\;}

\newcommand{\mystrut}{\rule{0em}{2.5ex}}
\newcommand{\ms}{\mystrut}
\newcommand{\cm}{\checkmark}

\newcommand{\Maps}[2]{\mathrm{Func}(#1,#2)}

\newcommand{\circuit}{(n,m,\ng,   \incomingleft,\incomingright,\function)}
\newcommand{\circuitt}{(n,m,\ng,  \incomingleft',\incomingright',\function)}
\newcommand{\circuittt}{(n,m,\ng',  \incomingleft',\incomingright',\function')}
\newcommand{\Circuit}{(n,m,\ng, \incomingleft,\incomingright,\function)}
\newcommand{\shortcircuit}{(n,m,\ng,   \incomingleft,\incomingright,\function)}

\newcommand{\topologicalcircuit}{(n,m,\ng, \incomingleft,\incomingright)}
\newcommand{\topologicalcircuitt}{(n,m,\ng, \incomingleft',\incomingright')}
\newcommand{\garbledcircuit}{(n,m,\ng, \incomingleft,\incomingright,\propagate)}
\newcommand{\garbledcircuitt}{(n,m,\ng, \incomingleft',\incomingright',\propagate)}
\newcommand{\shortgarbledcircuit}{(n,m,\ng, \incomingleft,\incomingright,\propagate)}


\newcommand{\Topo}{\mathrm{Topo}}
\newcommand{\Size}{\mathrm{Size}}


\newcommand{\GARBLEE}{\GARBLE\,'}
\newcommand{\GARBLEElist}{(\Garble', \Enc', \Dec', \Eval\,', \eval\,')}

\newcommand{\EE}{\mathcal{E}}
\newcommand{\DD}{\mathcal{D}}
\newcommand{\EEinv}{{\mathcal{D}}}

\newcommand{\EEE}[4]{\EE(#1,#2,#3,#4)}
\newcommand{\eee}[4]{\EE_{#1\smidgee#2}^{#3}(#4)}

\newcommand{\DDD}[4]   {\EEinv(#1\,#2,#3,#4)}
\newcommand{\EEEinv}[4]{\EEinv(#1\,#2,#3,#4)}

\newcommand{\ddd}[4]{\EEinv_{#1\smidgee#2}^{#3}(#4)}


\newcommand{\Dyn}{!}
\newcommand{\Dynn}[1]{{#1}\Dyn}
\newcommand{\DY}{!!}
\newcommand{\DYY}[1]{{#1}\DY}
\newcommand{\Uni}[1]{{#1}_\$}

\newcommand{\ekw}{\mathtt{e}}
\newcommand{\dkw}{\mathtt{d}}

\newcommand{\Gb}{\mathsf{Gb}}
\newcommand{\En}{\mathsf{En}}
\newcommand{\De}{\mathsf{De}}
\newcommand{\Ev}{\mathsf{Ev}}
\newcommand{\ev}{\mathsf{ev}}

\newcommand{\xn}{\mathsf{n}}
\newcommand{\xm}{\mathsf{m}}
\newcommand{\fn}{{f.n}}
\newcommand{\fnn}[1]{{f_{#1}.n}}
\newcommand{\fm}{{f.m}}
\newcommand{\fprimen}{{f'\!.n}}
\newcommand{\fprimem}{{f'\!.m}}

\newcommand{\informationn}{\textsf{n}'}
\newcommand{\informationm}{\textsf{m}'} 

\newcommand{\Garble}{\mathsf{Gb}}
\newcommand{\GarbleSplit}{\overline{\mathsf{Gb}}}
\newcommand{\Eval}{\mathsf{Ev}}
\newcommand{\pEval}{\overline{\mathsf{Ev}}}
\newcommand{\CEval}{{\mathsf{Ev}}}
\newcommand{\eval}{\mathsf{ev}}    
\newcommand{\peval}{\overline{\mathsf{ev}}}
\newcommand{\ceval}{\mathsf{ev}}
\newcommand{\circeval}{\mathsf{ev}_{\mathsf{circ}}}
\newcommand{\Encode}{\mathsf{En}}
\newcommand{\AIKEncode}{\mathsf{Enc}}
\newcommand{\Decode}{\mathsf{De}}
\newcommand{\AIKDecode}{\mathsf{Dec}}
\newcommand{\AIKSim}{\mathsf{Sim}}
\newcommand{\Gen}{\mathsl{Gen}}
\newcommand{\Decc}[1]{\Dec(#1)}        

\newcommand{\evall}[2]{\eval_{#1}(#2)}

\newcommand{\Evall}[1]{{#1}_\Eval}
\newcommand{\Evale}[1]{\Eval(#1)}

\newcommand{\GARBLElist}{(\Garble,\allowbreak \Enc,\allowbreak\Dec,\allowbreak \Eval,\allowbreak \eval)}
\newcommand{\select}{\mathsl{select}}
\newcommand{\Standard}{\mathsf{Standardize}}
\newcommand{\Parse}{\mathsf{Parse}}

%\newcommand{\Evalt}{\mathsl{Eval}}
%\newcommand{\Ft}{\tilde{F}}

\newcommand{\interpret}{\phi}

\newcommand{\EVAL}{\mathsl{EVAL}}

\newcommand{\bprime}{b\,'}
\newcommand{\Func}{\mathrm{Func}}
\newcommand{\Funcc}[2]{\Func({#1},{#2})}
\newcommand{\Dom}[1]{\mathrm{Dom}({#1})}
%\newcommand{\Domain}{\mathrm{Dom}}
\newcommand{\Ran}[1]{\mathrm{Ran}({#1})}
%\newcommand{\Range}{\mathrm{Ran}}
\newcommand{\BadDom}{\mathrm{BadDom}}
\newcommand{\BadRan}{\mathrm{BadRan}}


\newcommand{\map}{\pi}

%\newcommand{\boldstar}{{\boldsymbol{*}}}
%\newcommand{\bboldstar}{^{\boldsymbol{\,!}}}
%\newcommand{\bboldstar}{{\boldsymbol{!}}}
\newcommand{\bboldstar}{!}

\newcommand{\Garblefont}[1]{\mathrm{#1}}
\newcommand{\OT}{\Garblefont{OT}}


\newcommand{\Adv}{\mathbf{Adv}}
\newcommand{\cpa}{\mathrm{prp}}
\newcommand{\kdm}{\mathrm{kdm}}
\newcommand{\rr}{\mathrm{rr}}
\newcommand{\fg}{\mathrm{fg}}

\newcommand{\hyp}{\text{-}}

\newcommand{\extend}[1]{\overline{#1}}

\newcommand{\xxxnotion}  {\mathrm{xxx}}
\newcommand{\xnotion}[1] {\mathrm{xxx.#1}}




\definecolor{light-gray}{gray}{0.85}
\newcommand{\highlight}[1]{\fcolorbox{light-gray}{light-gray}{#1}}


\newcommand{\lsb}{\mathrm{lsb}}
\newcommand{\msb}{\mathrm{msb}}

\newcommand{\Code}[1]{\langle{#1}\rangle}
\newcommand{\Pair}[2]{#1\|#2}
\newcommand{\Split}[2]{(#1,#2)}


\newcommand{\encode}[1]{\hat{#1}}
\newcommand{\adversary}{{\mathscr{A}}}
\newcommand{\adversaryI}{{\calI}}
\newcommand{\gadversary}[1]{{A}_{\mathrm{#1}}}
\newcommand{\PrivIndA}{\mathcal{A}}
\newcommand{\PrivSimA}{\mathcal{B}}
\newcommand{\PrivSimAC}{\mathcal{B}_{\GARBLE}}
\newcommand{\OblvIndA}{\mathcal{A}}
\newcommand{\OblvSimA}{\mathcal{B}}
\newcommand{\PfeSimA}{\mathcal{B}}
\newcommand{\OTSimA}{\mathcal{B}_{\OTscheme}}
\newcommand{\PrfA}{\mathcal{D}}
\newcommand{\aclass}[1]{\calA_{\mathrm{#1}}}

% \newcommand{\calY}{{\cal Y}}
\newcommand{\advice}{{\alpha}}
\newcommand{\elusive}[2]{\varepsilon(#1, #2)}
\newcommand{\LP}{\mathtt{Gb3}}
\newcommand{\newPi}{\mathrm{\Pi}}
%\newcommand{\Sim}{{\cal S}}
\newcommand{\Sim}{{\calS}}
\newcommand{\OTSim}{{\calS}_{\OTscheme}}
\newcommand{\GSim}{{\calS}_{\GARBLE}}
\newcommand{\sepSim}{,\,\Sim}

\newcommand{\Output}{\mathrm{Output}}
\newcommand{\sfe}{\mathrm{SFE}}
\newcommand{\pfe}{\mathrm{PFE}}
\newcommand{\bitsem}{\phi}
\newcommand{\otp}{P}  %\cal P
\newcommand{\OTC}{\mathrm{OTC}}
\newcommand{\OTM}{\mathrm{OTM}}  
%\newcommand{\OTIME}{\mathrm{OTIME}}
\newcommand{\OTIME}{\mathrm{OTP}}
\newcommand{\OTP}{\mathrm{OTP}}
\newcommand{\Compile}{\mathrm{Create}}  %Compile
\newcommand{\Create}{\mathrm{Create}}  %Compile
\newcommand{\Compute}{\mathrm{Evaluate}}
\newcommand{\Evaluate}{\mathrm{Evaluate}}
\newcommand{\Equal}{\mathtt{Equal}}
\newcommand{\fanout}{\nu}
\newcommand{\xx}{\mathsf{X}}
\newcommand{\yy}{\mathsf{Y}}
\newcommand{\PX}{\mathsf{X}} %P_\xx
\newcommand{\PY}{\mathsf{Y}} %P_\yy}
\newcommand{\cc}{\mathsf{F}} %was \f
\newcommand{\PC}{\cc}  %P_\cc
\newcommand{\ID}{\mathrm{ID}}
%Pseudocode
\newcommand{\constfont}[1]{{\textsf{#1}}}
\newcommand{\true} {{\constfont{true}}}
\newcommand{\false}{{\constfont{false}}}
\newcommand{\kwfont}{\bf}
\newcommand{\used}{\mathsl{used}}
\newcommand{\called}{\mathsl{called}}
\newcommand{\rand}{\mathsl{rnd}}

\newcommand{\ALGORITHM}{\mbox{\kwfont algorithm}\xspace}
\newcommand{\AND}{\mbox{\kwfont and}\xspace}
\newcommand{\NOT}{\mbox{\kwfont not}\xspace}
\newcommand{\AS}{\mbox{\kwfont as}\xspace}
\newcommand{\DO}{\mbox{\kwfont do}\xspace}
\newcommand{\IF}{\mbox{\kwfont if}\xspace}
\newcommand{\FI}{\mbox{\kwfont fi}\xspace}
\newcommand{\ELSE}{\mbox{\kwfont else}\xspace}
\newcommand{\ELSEIF}{\mbox{\kwfont elsif}\xspace}
\newcommand{\FOR}{\mbox{\kwfont for}\xspace}
\newcommand{\OD}{\mbox{\kwfont od}\xspace}
\newcommand{\OR}{\mbox{\kwfont or}\xspace}
\newcommand{\PARSE}{\mbox{\kwfont parse}\xspace}
\newcommand{\PROCEDURE}{\mbox{\kwfont proc}\xspace}
\newcommand{\PRIVATE}{\mbox{\kwfont private}\xspace}
\newcommand{\RETURN}{\mbox{\kwfont return}\xspace}
\newcommand{\SELECT}{\mbox{\kwfont select}\xspace}
\newcommand{\THEN}{\mbox{\kwfont then}\xspace}
\newcommand{\WHERE}{\mbox{\kwfont where}\xspace}

\newcommand{\TO}{\mbox{\kwfont to}\xspace}
\newcommand{\BAD}{\mathsf{BAD}}
\newcommand{\GameName}[1]{\mathrm{#1}}
\newcommand{\Exp}{\mathbf{E}} %expectation
\newcommand{\Bounded}{\mathrm{KDM}}
\newcommand{\equal}{\mathsl{eq}}
\newcommand{\Bad}{\mathrm{Bad}}
\newcommand{\Query}{\mathrm{Query}}
\newcommand{\RO}{\mathrm{RO}}
\newcommand{\bad}{\mathsf{bad}}
\newcommand{\setsbad}{\mbox{ sets }\mathsf{bad}}

\newcommand{\keys}{\mathsl{keys}}
\newcommand{\ptr}{\mathsl{keynum}}
\newcommand{\GameIndex}[1]{\mathrm{G}_{#1}}
\newcommand{\GameL}{\mathrm{L}}
\newcommand{\DKQuery}{\mathsc{Dk}}
\newcommand{\pos}{\mathsl{pos}}
\newcommand{\type}{\mathsl{last}}
\newcommand{\FN}{\iftwocol \else \footnotesize\fi}
\newcommand{\commentchar}{\mbox{/\hspace{-0.5ex}/}}
%\newcommand{\comment}[1]{\commentchar\textsl{{#1}}}




\newcommand{\ZF}{Z_{\mathtt{F}}}
\newcommand{\ZD}{Z_{\mathtt{d}}}
\newcommand{\LF}{\lambda_{\mathtt{F}}}
\newcommand{\LD}{\lambda_{\mathtt{d}}}
\newcommand{\LX}{\lambda_{\mathtt{x}}}
\newcommand{\GetLen}{\mathsf{GetLen}}



\newcommand{\invert}{\mathrm{Invert}}

\newenvironment{newitemize}{%
\begin{list}{$\bullet$}{\labelwidth=19pt%
\labelsep=7pt \leftmargin=26pt \topsep=3pt%
\setlength{\listparindent}{\saveparindent}%
\setlength{\parsep}{\saveparskip}%
\setlength{\itemsep}{3pt} }}{\end{list}}
\newcommand{\sideinformation}{side-information\xspace} % or side-information
\newcommand{\tbl}{\mathtt{Tbl}}
\newcommand{\blockcipher}{blockcipher\xspace}
\newcommand{\blockciphers}{blockciphers\xspace}
\newcommand{\blockcipherbased}{blockcipher-based\xspace}
\newcommand{\Blockcipher}{Blockcipher\xspace}
\newcommand{\projective}{projective\xspace} %conventional, componentwise, bitwise, token-based, ...
\newcommand{\const}{\mathsf{const}}
\newcommand{\HMAC}{\mathrm{HMAC}}
\newcommand{\SHA}{\mathrm{SHA}}
\newcommand{\HMACSHA}{\mathrm{HMAC\mbox{-}SHA}}
\newcommand{\Projective}{Projective\xspace}


%KMR's notation
\newcommand{\KMROut}{\mathsf{GbOut}}
\newcommand{\KMRIn}{\mathsf{GbIn}}
\newcommand{\KMRCirc}{\mathsf{GbCircuit}}
\newcommand{\hatC}{\hat{C}}
\newcommand{\hatX}{\hat{X}}
\newcommand{\hatY}{\hat{Y}}

%\newcommand{\scrK}{\mathscr{K}}  %security parameter
\newcommand{\SP}{\mathsf{K}}  %security parameter
\newcommand{\Cinv}{C^{-1}}
\newcommand{\invtopo}{M_{\mathrm{topo}}}
\newcommand{\invsize}{M_{\mathrm{size}}}
\newcommand{\Prot}{\Pi}
\newcommand{\OTProt}{\Prott{ot}}
\newcommand{\Prott}[1]{\Prot^{\mathrm{#1}}}
\newcommand{\OTev}{\evv{ot}}
\newcommand{\evv}[1]{\mathsf{ev}^{\mathrm{#1}}}
\newcommand{\PFEscheme}{\calF}
\newcommand{\OTscheme}{{\cal OT}}
\newcommand{\viewproc}{\mathsc{GetView}}
\newcommand{\View}{\mathsf{View}}
\newcommand{\outp}{\mathsf{Out}}
\newcommand{\conv}{\mathit{conv}}
\newcommand{\convot}{\mathit{conv}^{\mathrm{ot}}}
\newcommand{\viewot}{\mathit{view}^{\mathrm{ot}}}
\newcommand{\coinsot}{\omega_1^{\mathrm{ot}}}
\newcommand{\view}{\mathit{view}}
\newcommand{\Phiot}{\Phi_{\mathrm{ot}}}

\newcommand{\citee}[2]{\iftwocol\cite{#2} (#1)\else\cite[#1]{#2}\fi}
\newcommand{\BeginFiguree}{\iftwocol\begin{figure*}\else\begin{figure}[t!]\small \fi} 
\newcommand{\EndFiguree}{\iftwocol\vspace{-0.1in}\end{figure*}\else\end{figure}\fi} 
\newcommand{\BeginFigure}{\iftwocol\begin{figure}[t!]\else\begin{figure}[t!]\small \fi} 
\newcommand{\EndFigure}{\iftwocol\vspace{-0.1in}\end{figure}\else\end{figure}\fi} 
\newcommand{\Footnote}[1]{\ifnofootnotes\else\footnote{#1}\fi}

\iftwocol
\newcommand{\num}[1]{}
\else
\newcommand{\num}[1]{{#1}\>}
\fi

\renewcommand{\eqref}[1]{Eq.~(\ref{#1})}
\newcommand{\headorsub}[1] {\iftwocol\heading{{#1}.}\else\subsection{{#1}}\fi}
\newcommand{\headorsubb}[1]{\iftwocol\noskipheading{{#1}.}\else\subsection{{#1}}\fi}


\newcommand{\prvpsim}{\mathrm{prv.prom}}
\newcommand{\prvnpsim}{\mathrm{prv.nprom}}
\newcommand{\obvpsim}{\mathrm{obv.prom}}
\newcommand{\obvnpsim}{\mathrm{obv.nprom}}
\newcommand{\Keys}{\mathrm{Keys}}
\newcommand{\calO}{{\cal O}}

\newcommand{\calDD}{\mathscr{D}}
\newcommand{\calEE}{\mathscr{E}}
\newcommand{\calKK}{\mathscr{K}}
\newcommand{\clen}{\mathsl{clen}}

\newcommand{\goodkey}{K_{\SE}}
\newcommand{\smashh}{}   %wasn't really working reliably; often overwrites the tilde
\newcommand{\tildeL}    {{\smashh{\widetilde{L}}}}
\newcommand{\badkey}{\tildeK}
\newcommand{\tildeK}    {K_{\BB}}
\newcommand{\tildecalK} {\bits^{\BBkl}}
\newcommand{\tildecalE} {\BB.\mathsf{Enc}}
\newcommand{\tildecalD} {\BB.\mathsf{Dec}}
\newcommand{\tildePi}   {\BB}
\newcommand{\BBExt}{\BB.\mathsf{Ext}}
\newcommand{\tildecalEE}{{\smashh{\widetilde{\calEE}}}}
\newcommand{\tildecalDD}{{\smashh{\widetilde{\calDD}}}}
\newcommand{\tildeC}    {{\smashh{\widetilde{C}}}}

\newcommand{\tildeKa}    {{{\widetilde{K}}}}
\newcommand{\tildecalKa} {{{\widetilde{\calK}}}}
\newcommand{\tildecalEa} {{{\widetilde{\calE}}}}
\newcommand{\tildecalDa} {{{\widetilde{\calD}}}}
\newcommand{\tildePia}   {{{\widetilde{\Pi}}}}
\newcommand{\tildecalEEa}{{{\widetilde{\calEE}}}}
\newcommand{\tildecalDDa}{{{\widetilde{\calDD}}}}
\newcommand{\tildeCa}    {{{\widetilde{C}}}}

\newcommand{\advrandom}{\adv_{\$}}
\newcommand{\advnonce}{\adv_{1}}
\newcommand{\advcounter}{\adv_{\mathtt{+}}}
\newcommand{\state}{\sigma}

\newcommand{\bfM}{\boldsymbol{M}}
\newcommand{\bfC}{\boldsymbol{C}}
\newcommand{\bfA}{\boldsymbol{A}}

\newcommand{\adv}{{\mathscr{A}}}
% \newcommand{\BB}{\mathscr{B}}
\newcommand{\BA}{\mathscr{A}}
\newcommand{\users}{\mathscr{D}}
\newcommand{\prfadv}{\mathscr{F}}
\newcommand{\rsadv}{\mathscr{R}}
\newcommand{\real}{{\mathrm{real}}}
\newcommand{\fake}{{\mathrm{fake}}}
\newcommand{\subvert}{{\mathrm{subvert}}}
\newcommand{\calEEreal}{{\calEE_1}}
\newcommand{\calEEfake}{{\calEE_0}}
\newcommand{\calEEsubvert}{{\calEE_0}}

\newcommand{\gamePriv}{\mathrm{PRIV}}
\newcommand{\gameDetect}{\mathrm{SDET}}
\newcommand{\gameDetectt}{\mathrm{DET2}}
\newcommand{\gameSurveillance}{\mathrm{SURV}}
\newcommand{\gameSurveillancee}{\mathrm{SURV2}}

\newcommand{\birep}[1]{\langle #1\rangle}

\newcommand{\gamesfontsize}{\small}
\newcommand{\ind}{\hspace*{10pt}}

\newcommand{\mpage}[2]{\begin{minipage}{#1\textwidth}
    #2 \end{minipage}}


\newcommand{\oneCol}[2]{
\begin{center}
        \framebox{
        \begin{tabular}{c@{\hspace*{.4em}}c}
        \begin{minipage}[t]{#1\textwidth}\setstretch{1.2}\gamesfontsize #2 \end{minipage}
        \end{tabular}
        }
\end{center}
}



\newcommand{\twoColsNoDivide}[4]{
\begin{center}
        \framebox{
        \begin{tabular}{c@{\hspace*{.4em}}c@{\hspace*{.4em}}c}
        \begin{minipage}[t]{#1\textwidth}\setstretch{1.2}\gamesfontsize #3 \end{minipage}
        &
        \begin{minipage}[t]{#2\textwidth}\setstretch{1.2}\gamesfontsize #4 \end{minipage}
        \end{tabular}
        }
\end{center}
}


\newcommand{\twoCols}[4]{
\begin{center}
        \framebox{
        \begin{tabular}{c@{\hspace*{.4em}}|c@{\hspace*{.4em}}c}
        \begin{minipage}[t]{#1\textwidth}\setstretch{1.2}\gamesfontsize #3 \end{minipage}
        &~
        \begin{minipage}[t]{#2\textwidth}\setstretch{1.2}\gamesfontsize #4 \end{minipage}
        \end{tabular}
        }
\end{center}
}

\newcommand{\twoColsPics}[4]{
\begin{center}
        \begin{tabular}{c@{\hspace*{.4em}}c@{\hspace*{.4em}}c}
        \begin{minipage}[t][][c]{#1\textwidth}\setstretch{1.2}\gamesfontsize #3 \end{minipage}
        &~
        \begin{minipage}[t][][c]{#2\textwidth}\setstretch{1.2}\gamesfontsize #4 \end{minipage}
        \end{tabular}
\end{center}
}

\newcommand{\twoColsTwoRows}[6]{
\begin{center}
        \framebox{
        \begin{tabular}{c@{\hspace*{.4em}}|c@{\hspace*{.4em}}c}
        \begin{minipage}[t]{#1\textwidth}\setstretch{1.2}\gamesfontsize #3 \end{minipage}
        &
        \begin{minipage}[t]{#2\textwidth}\setstretch{1.2}\gamesfontsize #4 \end{minipage}
        \\ \hline
        \begin{minipage}[t]{#1\textwidth}\setstretch{1.2}\gamesfontsize #5
        \end{minipage} &
        \begin{minipage}[t]{#2\textwidth}\setstretch{1.2}\gamesfontsize #6        
        \end{minipage}
        \end{tabular}
        }
\end{center}
}

\newcommand{\threeCols}[6]{
\begin{center}
        \framebox{
        \begin{tabular}{@{\hspace{-0.2em}}c@{\hspace{0.2em}}|@{\hspace{0.2em}}c@{\hspace{0.2em}}|@{\hspace{0.2em}}c@{\hspace{0.2em}}}
        \begin{minipage}[t]{#1\textwidth}\setstretch{1.1}\gamesfontsize #4
        \end{minipage} &
        \begin{minipage}[t]{#2\textwidth}\setstretch{1.1}\gamesfontsize #5
        \end{minipage} &
        \begin{minipage}[t]{#3\textwidth}\setstretch{1.1}\gamesfontsize #6
        \end{minipage}
        \end{tabular}
        }
\end{center}
}

\newcommand{\threeColstwoSplit}[6]{
\begin{center}
        \framebox{
        \begin{tabular}{@{\hspace{-0.2em}}c@{\hspace{0.2em}}|@{\hspace{0.2em}}c@{\hspace{0.2em}}@{\hspace{0.2em}}c@{\hspace{0.2em}}}
        \begin{minipage}[t]{#1\textwidth}\setstretch{1.1}\gamesfontsize #4
        \end{minipage} &
        \begin{minipage}[t]{#2\textwidth}\setstretch{1.1}\gamesfontsize #5
        \end{minipage} &
        \begin{minipage}[t]{#3\textwidth}\setstretch{1.1}\gamesfontsize #6
        \end{minipage}
        \end{tabular}
        }
\end{center}
}

\newcommand{\threeColsnoSplit}[6]{
\begin{center}
        \framebox{
        \begin{tabular}{@{\hspace{-0.2em}}c@{\hspace{0.2em}}c@{\hspace{0.2em}}@{\hspace{0.2em}}c@{\hspace{0.2em}}}
        \begin{minipage}[t]{#1\textwidth}\setstretch{1.1}\gamesfontsize #4
        \end{minipage} &
        \begin{minipage}[t]{#2\textwidth}\setstretch{1.1}\gamesfontsize #5
        \end{minipage} &
        \begin{minipage}[t]{#3\textwidth}\setstretch{1.1}\gamesfontsize #6
        \end{minipage}
        \end{tabular}
        }
\end{center}
}

\newcommand{\threeColstwoSplitR}[6]{
\begin{center}
        \framebox{
        \begin{tabular}{@{\hspace{-0.2em}}c@{\hspace{0.2em}}c@{\hspace{0.2em}}|@{\hspace{0.2em}}@{\hspace{0.2em}}c@{\hspace{0.2em}}}
        \begin{minipage}[t]{#1\textwidth}\setstretch{1.1}\gamesfontsize #4
        \end{minipage} &
        \begin{minipage}[t]{#2\textwidth}\setstretch{1.1}\gamesfontsize #5
        \end{minipage} &
        \begin{minipage}[t]{#3\textwidth}\setstretch{1.1}\gamesfontsize #6
        \end{minipage}
        \end{tabular}
        }
\end{center}
}

\newcommand{\threeColstwoSplitL}[6]{
\begin{center}
        \framebox{
        \begin{tabular}{@{\hspace{-0.2em}}c@{\hspace{0.2em}}|@{\hspace{0.2em}}c@{\hspace{0.2em}}@{\hspace{0.2em}}c@{\hspace{0.2em}}}
        \begin{minipage}[t]{#1\textwidth}\setstretch{1.1}\gamesfontsize #4
        \end{minipage} &
        \begin{minipage}[t]{#2\textwidth}\setstretch{1.1}\gamesfontsize #5
        \end{minipage} &
        \begin{minipage}[t]{#3\textwidth}\setstretch{1.1}\gamesfontsize #6
        \end{minipage}
        \end{tabular}
        }
\end{center}
}

\newcommand{\threeColsSplit}[6]{
\begin{center}
        \framebox{
        \begin{tabular}{@{\hspace{-0.2em}}c@{\hspace{0.2em}}||@{\hspace{0.2em}}c@{\hspace{0.2em}}|@{\hspace{0.2em}}c@{\hspace{0.2em}}}
        \begin{minipage}[t]{#1\textwidth}\setstretch{1.1}\gamesfontsize #4
        \end{minipage} &
        \begin{minipage}[t]{#2\textwidth}\setstretch{1.1}\gamesfontsize #5
        \end{minipage} &
        \begin{minipage}[t]{#3\textwidth}\setstretch{1.1}\gamesfontsize #6
        \end{minipage}
        \end{tabular}
        }
\end{center}
}

\newcommand{\fourColstwoBox}[8]{
\begin{center}
        \framebox{
        \begin{tabular}{@{\hspace{-0.2em}}c@{\hspace{0.2em}}c@{\hspace{0.2em}}|@{\hspace{0.2em}}c@{\hspace{0.2em}}c@{\hspace{0.2em}}}
        \begin{minipage}[t]{#1\textwidth}\setstretch{1.1}\gamesfontsize #5
        \end{minipage} &
        \begin{minipage}[t]{#2\textwidth}\setstretch{1.1}\gamesfontsize #6
        \end{minipage} &
        \begin{minipage}[t]{#3\textwidth}\setstretch{1.1}\gamesfontsize #7
        \end{minipage} &
        \begin{minipage}[t]{#4\textwidth}\setstretch{1.1}\gamesfontsize #8
        \end{minipage}
        \end{tabular}
        }
\end{center}
}

\newcommand{\threeColsNoBox}[3]{
\begin{center}\begin{tabular}{c|c|c}
\begin{minipage}[t]{1in}\begin{tabbing}
12\=12\=12\=\kill
#1
\end{tabbing}\end{minipage} &
\begin{minipage}[t]{1in}\begin{tabbing}
12\=12\=12\=\kill
#2
\end{tabbing}\end{minipage} &
\begin{minipage}[t]{1in}\begin{tabbing}
12\=12\=12\=\kill
#3
\end{tabbing}\end{minipage} 
\end{tabular}\end{center}}



\newcommand{\twoColsNoBox}[2]{
\begin{center}\begin{tabular}{c|c}
\begin{minipage}[t]{1in}\begin{tabbing}
12\=12\=12\=\kill
#1
\end{tabbing}\end{minipage} &
~~\begin{minipage}[t]{1in}\begin{tabbing}
12\=12\=12\=\kill
#2
\end{tabbing}\end{minipage} 
\end{tabular}\end{center}}


\newcommand{\twoColsNoBoxNoDivide}[2]{
\begin{center}\begin{tabular}{ccc}
\begin{minipage}[t]{1in}\begin{tabbing}
12\=12\=12\=\kill
#1
\end{tabbing}\end{minipage} & \hspace{10pt} &
\begin{minipage}[t]{1in}\begin{tabbing}
12\=12\=12\=\kill
#2
\end{tabbing}\end{minipage} 
\end{tabular}\end{center}}


\newcommand{\rv}[1]{\mathsf{#1}}

\newcommand{\oneColNoBox}[1]{
\begin{center}\begin{tabular}{c}
\begin{minipage}[t]{1in}\begin{tabbing}
12\=12\=12\=\kill
#1
\end{tabbing}\end{minipage} 
\end{tabular}\end{center}}

\newcommand{\gamefont}[1]{\mathrm{#1}}
\newcommand{\procfont}[1]{\mathsc{#1}}
\newcommand{\Img}{\mathrm{Im}}
\newcommand{\SD}{\mathbf{SD}}

\newcommand{\gameG}{\gamefont{G}}


\newcommand{\SmpO}{\procfont{Smp}}
\newcommand{\EncOSim}{\procfont{EncSim}}
\newcommand{\KeyO}{\procfont{Key}}
\newcommand{\KeyOSim}{\procfont{KeySim}}
\newcommand{\LRO}{\procfont{LR}}
\newcommand{\LROSim}{\procfont{LRSim}}
\newcommand{\Fn}{\procfont{Fn}}
\newcommand{\RRO}{\procfont{RR}}


\newcommand{\survAdv}[2]{\Adv^{\mathrm{srv}}_{#1, #2}}
\newcommand{\zfunc}{\mathbf{0}}
\newcommand{\gp}{\mathbf{GP}}
\newcommand{\minentropy}{\mathbf{H}_{\infty}}
\newcommand{\rp}{\mathbf{RP}}

\newcommand{\Find}{\mathrm{Find}}
%\newcommand{\next}{\:;\:}

\newcommand{\prfch}{b_{\mathrm{prf}}}
\newcommand{\prfchp}{b_{\mathrm{prf}}'}
\newcommand{\sdetch}{b_{\mathrm{sdet}}}
\newcommand{\sdetchp}{b_{\mathrm{sdet}}'}
\newcommand{\distributionName}[1]{\underline{\textbf{Distribution} #1}\\[2pt]}
\newcommand{\indcpa}{\mathrm{indcpa}}
\newcommand{\indkdm}{\mathrm{indkdm}}
\newcommand{\tildef}{{\smashh{\widetilde{f}}}}
\newcommand{\kl}{\mathsf{kl}}
\newcommand{\rl}{\mathsf{rl}}
\newcommand{\bl}{\mathsf{bl}}
\newcommand{\Mdist}{\calM}
\newcommand{\gameKRec}{\mathrm{KR}}
\newcommand{\gameRS}{\mathrm{RS}}
\newcommand{\Guess}{\mathbf{Guess}}


%Julie Definitions
\newcommand{\dog}{silly}


%Joseph Definitions
\newcommand{\cat}{not silly}

\newcommand{\rkf}{\Pi}
\newcommand{\rkF}[1]{\rkf\xor #1}

\newcommand{\encoding}[1]{\left\langle#1\right\rangle}

\newcommand{\tagg}{\mathrm{Tag}}
\newcommand{\tl}{\mathsf{tl}}
\newcommand{\lmax}{\mathsf{max}}
\newcommand{\SIV}{\textrm{SIV}}
\newcommand{\St}{\mathrm{St}}
\newcommand{\ctr}{\mathrm{ctr}}
\newcommand{\insecH}{\epsilon_\Hash}

%gameStuff
\newcommand{\proc}[1]{\underline{\textbf{procedure} #1}\\[2pt]}
\newcommand{\advName}[1]{\underline{\textbf{adversary} #1}\\[2pt]}
\newcommand{\gameName}[1]{\underline{\textbf{Game} #1}\\[2pt]}
\newcommand{\gameSpace}{\hfill\\[2pt]}


\newcommand{\init}[1]{\proc{Initialize(#1)}}
\newcommand{\finalize}[1]{\proc{Finalize(#1)}}

%Security Game Names
\newcommand{\gameAEF}{\GameName{AE5}}
\newcommand{\gameRKAEF}{\GameName{RK\text{-}AE5}}
\newcommand{\gamePRF}{\GameName{PRF}}
\newcommand{\gameINDRCPARK}{\GameName{RK\text{-}IND\$}}
\newcommand{\gameINDRCPA}{\GameName{IND\$}}
\newcommand{\gameINDR}{\GameName{IND\$}}
\newcommand{\gameAU}{\GameName{AU}}
\newcommand{\gameREAL}{\GameName{REAL}}
\newcommand{\gameRAND}{\GameName{RAND}}
\newcommand{\gameLOSSY}{\GameName{LOSSY}}
\newcommand{\gameDLOG}{\GameName{DLOG}}
\newcommand{\gameDH}{\GameName{DH}}
\newcommand{\gameDETDH}{\GameName{DETDH}}
\newcommand{\gameSURV}{\GameName{SURV}}
\newcommand{\gameDETECT}{\GameName{DETECT}}

%Primitives
\newcommand{\SE}{\mathsf{AE}}
\newcommand{\SEKey}{K_{\SE}}
\newcommand{\SEkl}{\SE.\mathsf{kl}}
\newcommand{\SEcl}{\SE.\mathsf{cl}}
\newcommand{\SEtl}{\SE.\mathsf{tl}}
\newcommand{\SEnl}{\SE.\mathsf{nl}}
\newcommand{\SEsl}{\SE.\mathsf{sl}}
\newcommand{\SEKg}{\SE.\mathsf{Kg}}
\newcommand{\SEEnc}{\SE.\mathsf{Enc}}
\newcommand{\SEDec}{\SE.\mathsf{Dec}}
\newcommand{\AEEnc}{\SE.\mathsf{Enc}}
\newcommand{\AEDec}{\SE.\mathsf{Dec}}
\newcommand{\SECon}{\SE.\mathsf{Con}}

\newcommand{\ELF}{\mathsf{ELF}}
\newcommand{\ELFGen}{\ELF.\mathsf{Gen}}
\newcommand{\ELFEv}{\ELF.\mathsf{Ev}}
\newcommand{\ek}{ek}
\newcommand{\EGen}{\mathtt{ELF}.\mathsf{Gen}}

\newcommand{\E}{\mathsf{E}}
\newcommand{\EKey}{K_{\E}}
\newcommand{\EEnc}{\E.\mathsf{Enc}}
\newcommand{\EDec}{\E.\mathsf{Dec}}
\newcommand{\Enl}{\E.\mathsf{nl}}
\newcommand{\ECon}{\E.\mathsf{Con}}

\newcommand{\J}{\mathrm{J}}
\newcommand{\Jnl}{\J.\mathsf{nl}}
\newcommand{\JKey}{K_{\J}}
\newcommand{\JKg}{\J.\mathsf{Kg}}
\newcommand{\JEnc}{\J.\mathsf{Enc}}
\newcommand{\JDec}{\J.\mathsf{Dec}}
\newcommand{\JRd}{\J.\mathsf{Rd}}

\newcommand{\PRF}{\mathsf{F}}
\newcommand{\F}{\mathsf{F}}
\newcommand{\Fkl}{\F.\mathsf{kl}}
\newcommand{\Fnl}{\F.\mathsf{nl}}
\newcommand{\FKey}{K_{\F}}
\newcommand{\FKg}{\PRF.\mathsf{Kg}}
\newcommand{\FEv}{\PRF.\mathsf{Ev}}
\newcommand{\FRd}{\PRF.\mathsf{Rd}}
\newcommand{\FDom}{\PRF.\mathsf{D}}
\newcommand{\FRange}{\PRF.\mathsf{R}}
\newcommand{\FCon}{\PRF.\mathsf{Con}}
\newcommand{\fkey}{fk}

\newcommand{\Hash}{\mathsf{H}}
\newcommand{\Hol}{\Hash.\mathsf{ol}}
\newcommand{\Hkl}{\Hash.\mathsf{kl}}
\newcommand{\HKey}{K_{\Hash}}
\newcommand{\HKg}{\Hash.\mathsf{Kg}}
\newcommand{\HEv}{\Hash.\mathsf{Ev}}
\newcommand{\HDom}{\Hash.\mathsf{D}}
\newcommand{\HRange}{\Hash.\mathsf{R}}

\newcommand{\DS}{\mathsf{DS}}
\newcommand{\DSCon}{\DS.\mathsf{Con}}
\newcommand{\DSVf}{\DS.\mathsf{Vf}}
\newcommand{\DSSig}{\DS.\mathsf{Sig}}

%Advantages
\newcommand{\elf}{{\mathrm{elf}}}
\newcommand{\AdvELF}[2]{\Adv^{\elf}_{#1}(#2)}

%Oracles
\newcommand{\EncO}{\procfont{Enc}}
\newcommand{\DecO}{\procfont{Dec}}
\newcommand{\FnO}{\procfont{Fn}}
\newcommand{\EncSim}{\procfont{EncSim}}
\newcommand{\DecSim}{\procfont{DecSim}}
\newcommand{\FnSim}{\procfont{FnSim}}
\newcommand{\EvO}{\procfont{Ev}}

%Adversaries
\newcommand{\advA}{\mathcal{A}}
\newcommand{\advASE}{\advA_{\SE}}
\newcommand{\advB}{\mathcal{B}}
\newcommand{\advC}{\mathcal{C}}

%Proof Games
\newcommand{\Gm}{\GameName{G}}
\newcommand{\GmA}{\GameName{A}}
\newcommand{\GmB}{\GameName{B}}
\newcommand{\GmH}{\GameName{H}}
\newcommand{\GmI}{\GameName{I}}


\newcommand{\negl}{\mathbf{negl}}
\DeclareMathOperator*{\argmax}{arg\,max}
